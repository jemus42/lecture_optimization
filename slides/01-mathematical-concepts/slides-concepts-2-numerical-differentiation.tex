%<<setup-child, include = FALSE>>=
%library(knitr)
%library(microbenchmark)
%library(snow)
%library(colorspace)
%library(grid)
%library(gridExtra)
%library(dplyr)
%library(ggplot2)
%library(latex2exp)

%set_parent("../style/preamble.Rnw")

%pi = base::pi

%theme_set(theme_bw())

%source("rsrc/functions.R")
%@



\documentclass[11pt,compress,t,notes=noshow]{beamer}
\usepackage[]{graphicx}
\usepackage[]{color}
% maxwidth is the original width if it is less than linewidth
% otherwise use linewidth (to make sure the graphics do not exceed the margin)
\makeatletter
\def\maxwidth{ %
  \ifdim\Gin@nat@width>\linewidth
    \linewidth
  \else
    \Gin@nat@width
  \fi
}
\makeatother

\definecolor{fgcolor}{rgb}{0.345, 0.345, 0.345}
\newcommand{\hlnum}[1]{\textcolor[rgb]{0.686,0.059,0.569}{#1}}%
\newcommand{\hlstr}[1]{\textcolor[rgb]{0.192,0.494,0.8}{#1}}%
\newcommand{\hlcom}[1]{\textcolor[rgb]{0.678,0.584,0.686}{\textit{#1}}}%
\newcommand{\hlopt}[1]{\textcolor[rgb]{0,0,0}{#1}}%
\newcommand{\hlstd}[1]{\textcolor[rgb]{0.345,0.345,0.345}{#1}}%
\newcommand{\hlkwa}[1]{\textcolor[rgb]{0.161,0.373,0.58}{\textbf{#1}}}%
\newcommand{\hlkwb}[1]{\textcolor[rgb]{0.69,0.353,0.396}{#1}}%
\newcommand{\hlkwc}[1]{\textcolor[rgb]{0.333,0.667,0.333}{#1}}%
\newcommand{\hlkwd}[1]{\textcolor[rgb]{0.737,0.353,0.396}{\textbf{#1}}}%
\let\hlipl\hlkwb

\usepackage{framed}
\makeatletter
\newenvironment{kframe}{%
 \def\at@end@of@kframe{}%
 \ifinner\ifhmode%
  \def\at@end@of@kframe{\end{minipage}}%
  \begin{minipage}{\columnwidth}%
 \fi\fi%
 \def\FrameCommand##1{\hskip\@totalleftmargin \hskip-\fboxsep
 \colorbox{shadecolor}{##1}\hskip-\fboxsep
     % There is no \\@totalrightmargin, so:
     \hskip-\linewidth \hskip-\@totalleftmargin \hskip\columnwidth}%
 \MakeFramed {\advance\hsize-\width
   \@totalleftmargin\z@ \linewidth\hsize
   \@setminipage}}%
 {\par\unskip\endMakeFramed%
 \at@end@of@kframe}
\makeatother

\definecolor{shadecolor}{rgb}{.97, .97, .97}
\definecolor{messagecolor}{rgb}{0, 0, 0}
\definecolor{warningcolor}{rgb}{1, 0, 1}
\definecolor{errorcolor}{rgb}{1, 0, 0}
\definecolor{code}{rgb}{0.97, 0.96, 1.0}
\newenvironment{knitrout}{}{} % an empty environment to be redefined in TeX

\usepackage{alltt}

\usepackage[utf8]{inputenc}
\usepackage[ngerman]{babel}
\usepackage{dsfont}
\usepackage{verbatim}
\usepackage{amsmath}
\usepackage{amsfonts}
\usepackage{mathtools}
\usepackage{csquotes}
\usepackage{cmbright}
\usepackage{multirow}
\usepackage{longtable}
\usepackage{enumerate}
\usepackage[absolute,overlay]{textpos}
\usepackage{psfrag}
\usepackage{algorithm}
\usepackage{algpseudocode}
\usepackage{eqnarray}
\usepackage{bytefield}
\usepackage{animate}
\usepackage{tikz}
\usetikzlibrary{shapes,matrix,positioning,chains,arrows,shadows,decorations.pathmorphing,fit,backgrounds}
\usepackage{adjustbox}
\usepackage{colortbl}
\usepackage{tabularx} % for tables (incl. \hline)
\usepackage{arydshln} % Load after array, longtable, colortab and/or colortbl , otherwise problems with \hline in tabular env
\usepackage{etex} %increase registers for \dimenS to more than 256, otherwise we get "No room for a new \dimen"
\usepackage{graphicx}
\usepackage{booktabs} %used in epr lectures
\usepackage{bm} % bold greek letters
\usepackage{hyperref} % url citing
\usepackage{blkarray} % block arrays
\usepackage{listings} % block of code
\usepackage{xcolor} %colored math symbols
\usepackage{pgffor}
\usepackage{verbatimbox}
\usepackage{xcolor}

%some colors
\definecolor{checkgreen}{HTML}{18A126}
\definecolor{errorred}{HTML}{FF0000}
\definecolor{blockbg}{HTML}{F7F7F7}
\definecolor{gray}{HTML}{A0A0A0}


% basic latex stuff
\newcommand{\col}{\par\colorbox{code}{\parbox{\textwidth}{\theverbbox}}\par}
\newcommand{\eg}{e.\,g.\xspace} %for example
\newcommand{\ie}{i.\,e.\xspace} %that is to say...
\newcommand{\pkg}[1]{{\fontseries{b}\selectfont #1}} %fontstyle for R packages
\newcommand{\lz}{\vspace{0.5cm}} %vertical space
\newcommand{\oneliner}[1] % Oneliner for important statements
{\begin{block}{}\begin{center}\begin{Large}#1\end{Large}\end{center}\end{block}}
\def\SpAr{\quad \Rightarrow \quad}



%new environments

\newenvironment{vbframe}  %frame with breaks and verbatim
{
 \begin{frame}[containsverbatim,allowframebreaks]
}
{
\end{frame}
}

\newenvironment{vframe}  %frame with verbatim without breaks (to avoid numbering one slided frames)
{
 \begin{frame}[containsverbatim]
}
{
\end{frame}
}

\newenvironment{blocki}[1]   % itemize block
{
 \begin{block}{#1}\begin{itemize}
}
{
\end{itemize}\end{block}
}

\newenvironment{fragileframe}[2]{  %fragile frame with framebreaks
\begin{frame}[allowframebreaks, fragile, environment = fragileframe]
\frametitle{#1}
#2}
{\end{frame}}


\newcommand{\myframe}[2]{  %short for frame with framebreaks
\begin{frame}[allowframebreaks]
\frametitle{#1}
#2
\end{frame}}



% ???? remove this
% \newcommand{\LS}{\mathfrak{L}}
% \newcommand{\TS}{\mathfrak{T}}
% \newcommand{\bmat}{\begin{pmatrix}}
% \newcommand{\emat}{\end{pmatrix}}
% \newcommand{\const}{\mathop{const}}
% \newcommand{\dist}{\operatorname{dist}}
% \newcommand{\D}{\displaystyle}
%\newcommand{\op}[1]{\operatorname{#1}}

%\usetheme{../style/lmu-lecture}
\usepackage{../style/lmu-lecture}
\let\code=\texttt
\let\proglang=\textsf

\setkeys{Gin}{width=0.9\textwidth}








\usepackage{tikz}
\usetikzlibrary{shapes,arrows,snakes, calc}

% Define block styles
\tikzstyle{decision} = [diamond, draw, text width=6em, text badly centered, node distance=4cm, inner sep=0pt]
\tikzstyle{decision2} = [diamond, draw, fill=customgreen!35, text width=6em, text badly centered, node distance=4cm, inner sep=0pt]

\tikzstyle{block} = [rectangle, draw, text width=14em, text centered, rounded corners, node distance=3cm, minimum height=4em]
\tikzstyle{line} = [draw, -latex']
\tikzstyle{cloud} = [draw, ellipse, node distance=3cm, minimum height=2em]

\title{Optimization (CIM1)}
\author{Bernd Bischl\\ \and
Christian M\"uller}

\institute{Institut f\"ur Statistik -- LMU M\"unchen}
\date{WS 2021/2022}

\setbeamertemplate{frametitle}{\expandafter\uppercase\expandafter\insertframetitle}

%% Get lecture number from currente directory
%% Add '\lecturechapter{lecture_nr}{[NAME LECTURE]}' at beginning of .Rnw files


\IfFileExists{upquote.sty}{\usepackage{upquote}}{}




% math spaces
\ifdefined\N                                                                
\renewcommand{\N}{\mathds{N}} % N, naturals
\else \newcommand{\N}{\mathds{N}} \fi 
\newcommand{\Z}{\mathds{Z}} % Z, integers
\newcommand{\Q}{\mathds{Q}} % Q, rationals
\newcommand{\R}{\mathds{R}} % R, reals
\ifdefined\C 
  \renewcommand{\C}{\mathds{C}} % C, complex
\else \newcommand{\C}{\mathds{C}} \fi
\newcommand{\continuous}{\mathcal{C}} % C, space of continuous functions
\newcommand{\M}{\mathcal{M}} % machine numbers
\newcommand{\epsm}{\epsilon_m} % maximum error

% counting / finite sets
\newcommand{\setzo}{\{0, 1\}} % set 0, 1
\newcommand{\setmp}{\{-1, +1\}} % set -1, 1
\newcommand{\unitint}{[0, 1]} % unit interval

% basic math stuff
\newcommand{\xt}{\tilde x} % x tilde
\newcommand{\argmax}{\operatorname{arg\,max}} % argmax
\newcommand{\argmin}{\operatorname{arg\,min}} % argmin
\newcommand{\argminlim}{\mathop{\mathrm{arg\,min}}\limits} % argmax with limits
\newcommand{\argmaxlim}{\mathop{\mathrm{arg\,max}}\limits} % argmin with limits  
\newcommand{\sign}{\operatorname{sign}} % sign, signum
\newcommand{\I}{\mathbb{I}} % I, indicator
\newcommand{\order}{\mathcal{O}} % O, order
\newcommand{\pd}[2]{\frac{\partial{#1}}{\partial #2}} % partial derivative
\newcommand{\floorlr}[1]{\left\lfloor #1 \right\rfloor} % floor
\newcommand{\ceillr}[1]{\left\lceil #1 \right\rceil} % ceiling

% sums and products
\newcommand{\sumin}{\sum\limits_{i=1}^n} % summation from i=1 to n
\newcommand{\sumim}{\sum\limits_{i=1}^m} % summation from i=1 to m
\newcommand{\sumjn}{\sum\limits_{j=1}^n} % summation from j=1 to p
\newcommand{\sumjp}{\sum\limits_{j=1}^p} % summation from j=1 to p
\newcommand{\sumik}{\sum\limits_{i=1}^k} % summation from i=1 to k
\newcommand{\sumkg}{\sum\limits_{k=1}^g} % summation from k=1 to g
\newcommand{\sumjg}{\sum\limits_{j=1}^g} % summation from j=1 to g
\newcommand{\meanin}{\frac{1}{n} \sum\limits_{i=1}^n} % mean from i=1 to n
\newcommand{\meanim}{\frac{1}{m} \sum\limits_{i=1}^m} % mean from i=1 to n
\newcommand{\meankg}{\frac{1}{g} \sum\limits_{k=1}^g} % mean from k=1 to g
\newcommand{\prodin}{\prod\limits_{i=1}^n} % product from i=1 to n
\newcommand{\prodkg}{\prod\limits_{k=1}^g} % product from k=1 to g
\newcommand{\prodjp}{\prod\limits_{j=1}^p} % product from j=1 to p

% linear algebra
\newcommand{\one}{\boldsymbol{1}} % 1, unitvector
\newcommand{\zero}{\mathbf{0}} % 0-vector
\newcommand{\id}{\boldsymbol{I}} % I, identity
\newcommand{\diag}{\operatorname{diag}} % diag, diagonal
\newcommand{\trace}{\operatorname{tr}} % tr, trace
\newcommand{\spn}{\operatorname{span}} % span
\newcommand{\scp}[2]{\left\langle #1, #2 \right\rangle} % <.,.>, scalarproduct
\newcommand{\mat}[1]{\begin{pmatrix} #1 \end{pmatrix}} % short pmatrix command
\newcommand{\Amat}{\mathbf{A}} % matrix A
\newcommand{\Deltab}{\mathbf{\Delta}} % error term for vectors

% basic probability + stats
\renewcommand{\P}{\mathds{P}} % P, probability
\newcommand{\E}{\mathds{E}} % E, expectation
\newcommand{\var}{\mathsf{Var}} % Var, variance
\newcommand{\cov}{\mathsf{Cov}} % Cov, covariance
\newcommand{\corr}{\mathsf{Corr}} % Corr, correlation
\newcommand{\normal}{\mathcal{N}} % N of the normal distribution
\newcommand{\iid}{\overset{i.i.d}{\sim}} % dist with i.i.d superscript
\newcommand{\distas}[1]{\overset{#1}{\sim}} % ... is distributed as ...

% machine learning
\newcommand{\Xspace}{\mathcal{X}} % X, input space
\newcommand{\Yspace}{\mathcal{Y}} % Y, output space
\newcommand{\nset}{\{1, \ldots, n\}} % set from 1 to n
\newcommand{\pset}{\{1, \ldots, p\}} % set from 1 to p
\newcommand{\gset}{\{1, \ldots, g\}} % set from 1 to g
\newcommand{\Pxy}{\mathbb{P}_{xy}} % P_xy
\newcommand{\Exy}{\mathbb{E}_{xy}} % E_xy: Expectation over random variables xy
\newcommand{\xv}{\mathbf{x}} % vector x (bold)
\newcommand{\xtil}{\tilde{\mathbf{x}}} % vector x-tilde (bold)
\newcommand{\yv}{\mathbf{y}} % vector y (bold)
\newcommand{\xy}{(\xv, y)} % observation (x, y)
\newcommand{\xvec}{\left(x_1, \ldots, x_p\right)^\top} % (x1, ..., xp) 
\newcommand{\Xmat}{\mathbf{X}} % Design matrix
\newcommand{\allDatasets}{\mathds{D}} % The set of all datasets
\newcommand{\allDatasetsn}{\mathds{D}_n}  % The set of all datasets of size n 
\newcommand{\D}{\mathcal{D}} % D, data
\newcommand{\Dn}{\D_n} % D_n, data of size n
\newcommand{\Dtrain}{\mathcal{D}_{\text{train}}} % D_train, training set
\newcommand{\Dtest}{\mathcal{D}_{\text{test}}} % D_test, test set
\newcommand{\xyi}[1][i]{\left(\xv^{(#1)}, y^{(#1)}\right)} % (x^i, y^i), i-th observation
\newcommand{\Dset}{\left( \xyi[1], \ldots, \xyi[n]\right)} % {(x1,y1)), ..., (xn,yn)}, data
\newcommand{\defAllDatasetsn}{(\Xspace \times \Yspace)^n} % Def. of the set of all datasets of size n 
\newcommand{\defAllDatasets}{\bigcup_{n \in \N}(\Xspace \times \Yspace)^n} % Def. of the set of all datasets 
\newcommand{\xdat}{\left\{ \xv^{(1)}, \ldots, \xv^{(n)}\right\}} % {x1, ..., xn}, input data
\newcommand{\ydat}{\left\{ \yv^{(1)}, \ldots, \yv^{(n)}\right\}} % {y1, ..., yn}, input data
\newcommand{\yvec}{\left(y^{(1)}, \hdots, y^{(n)}\right)^\top} % (y1, ..., yn), vector of outcomes
\renewcommand{\xi}[1][i]{\xv^{(#1)}} % x^i, i-th observed value of x
\newcommand{\yi}[1][i]{y^{(#1)}} % y^i, i-th observed value of y 
\newcommand{\xivec}{\left(x^{(i)}_1, \ldots, x^{(i)}_p\right)^\top} % (x1^i, ..., xp^i), i-th observation vector
\newcommand{\xj}{\xv_j} % x_j, j-th feature
\newcommand{\xjvec}{\left(x^{(1)}_j, \ldots, x^{(n)}_j\right)^\top} % (x^1_j, ..., x^n_j), j-th feature vector
\newcommand{\phiv}{\mathbf{\phi}} % Basis transformation function phi
\newcommand{\phixi}{\mathbf{\phi}^{(i)}} % Basis transformation of xi: phi^i := phi(xi)

%%%%%% ml - models general
\newcommand{\lamv}{\bm{\lambda}} % lambda vector, hyperconfiguration vector
\newcommand{\Lam}{\bm{\Lambda}}	 % Lambda, space of all hpos
% Inducer / Inducing algorithm
\newcommand{\preimageInducer}{\left(\defAllDatasets\right)\times\Lam} % Set of all datasets times the hyperparameter space
\newcommand{\preimageInducerShort}{\allDatasets\times\Lam} % Set of all datasets times the hyperparameter space
% Inducer / Inducing algorithm
\newcommand{\ind}{\mathcal{I}} % Inducer, inducing algorithm, learning algorithm 

% continuous prediction function f
\newcommand{\ftrue}{f_{\text{true}}}  % True underlying function (if a statistical model is assumed)
\newcommand{\ftruex}{\ftrue(\xv)} % True underlying function (if a statistical model is assumed)
\newcommand{\fx}{f(\xv)} % f(x), continuous prediction function
\newcommand{\fdomains}{f: \Xspace \rightarrow \R^g} % f with domain and co-domain
\newcommand{\Hspace}{\mathcal{H}} % hypothesis space where f is from
\newcommand{\fbayes}{f^{\ast}} % Bayes-optimal model
\newcommand{\fxbayes}{f^{\ast}(\xv)} % Bayes-optimal model
\newcommand{\fkx}[1][k]{f_{#1}(\xv)} % f_j(x), discriminant component function
\newcommand{\fh}{\hat{f}} % f hat, estimated prediction function
\newcommand{\fxh}{\fh(\xv)} % fhat(x)
\newcommand{\fxt}{f(\xv ~|~ \thetab)} % f(x | theta)
\newcommand{\fxi}{f\left(\xv^{(i)}\right)} % f(x^(i))
\newcommand{\fxih}{\hat{f}\left(\xv^{(i)}\right)} % f(x^(i))
\newcommand{\fxit}{f\left(\xv^{(i)} ~|~ \thetab\right)} % f(x^(i) | theta)
\newcommand{\fhD}{\fh_{\D}} % fhat_D, estimate of f based on D
\newcommand{\fhDtrain}{\fh_{\Dtrain}} % fhat_Dtrain, estimate of f based on D
\newcommand{\fhDnlam}{\fh_{\Dn, \lamv}} %model learned on Dn with hp lambda
\newcommand{\fhDlam}{\fh_{\D, \lamv}} %model learned on D with hp lambda
\newcommand{\fhDnlams}{\fh_{\Dn, \lamv^\ast}} %model learned on Dn with optimal hp lambda 
\newcommand{\fhDlams}{\fh_{\D, \lamv^\ast}} %model learned on D with optimal hp lambda 

% discrete prediction function h
\newcommand{\hx}{h(\xv)} % h(x), discrete prediction function
\newcommand{\hh}{\hat{h}} % h hat
\newcommand{\hxh}{\hat{h}(\xv)} % hhat(x)
\newcommand{\hxt}{h(\xv | \thetab)} % h(x | theta)
\newcommand{\hxi}{h\left(\xi\right)} % h(x^(i))
\newcommand{\hxit}{h\left(\xi ~|~ \thetab\right)} % h(x^(i) | theta)
\newcommand{\hbayes}{h^{\ast}} % Bayes-optimal classification model
\newcommand{\hxbayes}{h^{\ast}(\xv)} % Bayes-optimal classification model

% yhat
\newcommand{\yh}{\hat{y}} % yhat for prediction of target
\newcommand{\yih}{\hat{y}^{(i)}} % yhat^(i) for prediction of ith targiet
\newcommand{\resi}{\yi- \yih}

% theta
\newcommand{\thetah}{\hat{\theta}} % theta hat
\newcommand{\thetab}{\bm{\theta}} % theta vector
\newcommand{\thetabh}{\bm{\hat\theta}} % theta vector hat
\newcommand{\thetat}[1][t]{\thetab^{[#1]}} % theta^[t] in optimization
\newcommand{\thetatn}[1][t]{\thetab^{[#1 +1]}} % theta^[t+1] in optimization
\newcommand{\thetahDnlam}{\thetabh_{\Dn, \lamv}} %theta learned on Dn with hp lambda
\newcommand{\thetahDlam}{\thetabh_{\D, \lamv}} %theta learned on D with hp lambda
\newcommand{\mint}{\min_{\thetab \in \Theta}} % min problem theta
\newcommand{\argmint}{\argmin_{\thetab \in \Theta}} % argmin theta

% densities + probabilities
% pdf of x 
\newcommand{\pdf}{p} % p
\newcommand{\pdfx}{p(\xv)} % p(x)
\newcommand{\pixt}{\pi(\xv~|~ \thetab)} % pi(x|theta), pdf of x given theta
\newcommand{\pixit}[1][i]{\pi\left(\xi[#1] ~|~ \thetab\right)} % pi(x^i|theta), pdf of x given theta
\newcommand{\pixii}[1][i]{\pi\left(\xi[#1]\right)} % pi(x^i), pdf of i-th x 

% pdf of (x, y)
\newcommand{\pdfxy}{p(\xv,y)} % p(x, y)
\newcommand{\pdfxyt}{p(\xv, y ~|~ \thetab)} % p(x, y | theta)
\newcommand{\pdfxyit}{p\left(\xi, \yi ~|~ \thetab\right)} % p(x^(i), y^(i) | theta)

% pdf of x given y
\newcommand{\pdfxyk}[1][k]{p(\xv | y= #1)} % p(x | y = k)
\newcommand{\lpdfxyk}[1][k]{\log p(\xv | y= #1)} % log p(x | y = k)
\newcommand{\pdfxiyk}[1][k]{p\left(\xi | y= #1 \right)} % p(x^i | y = k)

% prior probabilities
\newcommand{\pik}[1][k]{\pi_{#1}} % pi_k, prior
\newcommand{\lpik}[1][k]{\log \pi_{#1}} % log pi_k, log of the prior
\newcommand{\pit}{\pi(\thetab)} % Prior probability of parameter theta

% posterior probabilities
\newcommand{\post}{\P(y = 1 ~|~ \xv)} % P(y = 1 | x), post. prob for y=1
\newcommand{\postk}[1][k]{\P(y = #1 ~|~ \xv)} % P(y = k | y), post. prob for y=k
\newcommand{\pidomains}{\pi: \Xspace \rightarrow \unitint} % pi with domain and co-domain
\newcommand{\pibayes}{\pi^{\ast}} % Bayes-optimal classification model
\newcommand{\pixbayes}{\pi^{\ast}(\xv)} % Bayes-optimal classification model
\newcommand{\pix}{\pi(\xv)} % pi(x), P(y = 1 | x)
\newcommand{\piv}{\bm{\pi}} % pi, bold, as vector
\newcommand{\pikx}[1][k]{\pi_{#1}(\xv)} % pi_k(x), P(y = k | x)
\newcommand{\pikxt}[1][k]{\pi_{#1}(\xv ~|~ \thetab)} % pi_k(x | theta), P(y = k | x, theta)
\newcommand{\pixh}{\hat \pi(\xv)} % pi(x) hat, P(y = 1 | x) hat
\newcommand{\pikxh}[1][k]{\hat \pi_{#1}(\xv)} % pi_k(x) hat, P(y = k | x) hat
\newcommand{\pixih}{\hat \pi(\xi)} % pi(x^(i)) with hat
\newcommand{\pikxih}[1][k]{\hat \pi_{#1}(\xi)} % pi_k(x^(i)) with hat
\newcommand{\pdfygxt}{p(y ~|~\xv, \thetab)} % p(y | x, theta)
\newcommand{\pdfyigxit}{p\left(\yi ~|~\xi, \thetab\right)} % p(y^i |x^i, theta)
\newcommand{\lpdfygxt}{\log \pdfygxt } % log p(y | x, theta)
\newcommand{\lpdfyigxit}{\log \pdfyigxit} % log p(y^i |x^i, theta)

% probababilistic
\newcommand{\bayesrulek}[1][k]{\frac{\P(\xv | y= #1) \P(y= #1)}{\P(\xv)}} % Bayes rule
\newcommand{\muk}{\bm{\mu_k}} % mean vector of class-k Gaussian (discr analysis) 

% residual and margin
\newcommand{\eps}{\epsilon} % residual, stochastic
\newcommand{\epsi}{\epsilon^{(i)}} % epsilon^i, residual, stochastic
\newcommand{\epsh}{\hat{\epsilon}} % residual, estimated
\newcommand{\yf}{y \fx} % y f(x), margin
\newcommand{\yfi}{\yi \fxi} % y^i f(x^i), margin
\newcommand{\Sigmah}{\hat \Sigma} % estimated covariance matrix
\newcommand{\Sigmahj}{\hat \Sigma_j} % estimated covariance matrix for the j-th class

% ml - loss, risk, likelihood
\newcommand{\Lyf}{L\left(y, f\right)} % L(y, f), loss function
\newcommand{\Lypi}{L\left(y, \pi\right)} % L(y, pi), loss function
\newcommand{\Lxy}{L\left(y, \fx\right)} % L(y, f(x)), loss function
\newcommand{\Lxyi}{L\left(\yi, \fxi\right)} % loss of observation
\newcommand{\Lxyt}{L\left(y, \fxt\right)} % loss with f parameterized
\newcommand{\Lxyit}{L\left(\yi, \fxit\right)} % loss of observation with f parameterized
\newcommand{\Lxym}{L\left(\yi, f\left(\bm{\tilde{x}}^{(i)} ~|~ \thetab\right)\right)} % loss of observation with f parameterized
\newcommand{\Lpixy}{L\left(y, \pix\right)} % loss in classification
\newcommand{\Lpiv}{L\left(y, \piv\right)} % loss in classification
\newcommand{\Lpixyi}{L\left(\yi, \pixii\right)} % loss of observation in classification
\newcommand{\Lpixyt}{L\left(y, \pixt\right)} % loss with pi parameterized
\newcommand{\Lpixyit}{L\left(\yi, \pixit\right)} % loss of observation with pi parameterized
\newcommand{\Lhxy}{L\left(y, \hx\right)} % L(y, h(x)), loss function on discrete classes
\newcommand{\Lr}{L\left(r\right)} % L(r), loss defined on residual (reg) / margin (classif)
\newcommand{\lone}{|y - \fx|} % L1 loss
\newcommand{\ltwo}{\left(y - \fx\right)^2} % L2 loss
\newcommand{\lbernoullimp}{\ln(1 + \exp(-y \cdot \fx))} % Bernoulli loss for -1, +1 encoding
\newcommand{\lbernoullizo}{- y \cdot \fx + \log(1 + \exp(\fx))} % Bernoulli loss for 0, 1 encoding
\newcommand{\lcrossent}{- y \log \left(\pix\right) - (1 - y) \log \left(1 - \pix\right)} % cross-entropy loss
\newcommand{\lbrier}{\left(\pix - y \right)^2} % Brier score
\newcommand{\risk}{\mathcal{R}} % R, risk
\newcommand{\riskbayes}{\mathcal{R}^\ast}
\newcommand{\riskf}{\risk(f)} % R(f), risk
\newcommand{\riskdef}{\E_{y|\xv}\left(\Lxy \right)} % risk def (expected loss)
\newcommand{\riskt}{\mathcal{R}(\thetab)} % R(theta), risk
\newcommand{\riske}{\mathcal{R}_{\text{emp}}} % R_emp, empirical risk w/o factor 1 / n
\newcommand{\riskeb}{\bar{\mathcal{R}}_{\text{emp}}} % R_emp, empirical risk w/ factor 1 / n
\newcommand{\riskef}{\riske(f)} % R_emp(f)
\newcommand{\risket}{\mathcal{R}_{\text{emp}}(\thetab)} % R_emp(theta)
\newcommand{\riskr}{\mathcal{R}_{\text{reg}}} % R_reg, regularized risk
\newcommand{\riskrt}{\mathcal{R}_{\text{reg}}(\thetab)} % R_reg(theta)
\newcommand{\riskrf}{\riskr(f)} % R_reg(f)
\newcommand{\riskrth}{\hat{\mathcal{R}}_{\text{reg}}(\thetab)} % hat R_reg(theta)
\newcommand{\risketh}{\hat{\mathcal{R}}_{\text{emp}}(\thetab)} % hat R_emp(theta)
\newcommand{\LL}{\mathcal{L}} % L, likelihood
\newcommand{\LLt}{\mathcal{L}(\thetab)} % L(theta), likelihood
\newcommand{\LLtx}{\mathcal{L}(\thetab | \xv)} % L(theta|x), likelihood
\newcommand{\logl}{\ell} % l, log-likelihood
\newcommand{\loglt}{\logl(\thetab)} % l(theta), log-likelihood
\newcommand{\logltx}{\logl(\thetab | \xv)} % l(theta|x), log-likelihood
\newcommand{\errtrain}{\text{err}_{\text{train}}} % training error
\newcommand{\errtest}{\text{err}_{\text{test}}} % test error
\newcommand{\errexp}{\overline{\text{err}_{\text{test}}}} % avg training error

% lm
\newcommand{\thx}{\thetab^\top \xv} % linear model
\newcommand{\olsest}{(\Xmat^\top \Xmat)^{-1} \Xmat^\top \yv} % OLS estimator in LM 


%\usepackage{animate} % only use if you want the animation for Taylor2D

\begin{document}

\lecturechapter{2}{Numerical Differentiation}
\lecture{Optimization}

\begin{vbframe}{Differentiation}

  We consider a function $f: \R^d \rightarrow \R$. We have seen that for proving necessary and sufficient conditions for optima, we need to compute a function's derivatives. 
  
  \lz 
  
  We distinguish between: 
  
  \begin{itemize}
  \item Symbolic differentiation: Exact handling of mathematical expressions
  \item Numerical differentiation: Approximative calculation of the derivative
  \end{itemize}
  
  Numerical differentiation is often necessary if the derivative is not given, cannot be calculated or if the function itself is only indirectly available (e.g. via measured values).
  
  \end{vbframe}
  
  \begin{vbframe}{Numerical differentiation}
  \begin{itemize}
  \item Approximation of differentiation $\frac{\partial f}{\partial
  x_{i}}$:
  \begin{itemize}
    \item Newton's difference quotient for $\eps > 0$: 
    $$
    D_\xb(\epsilon) = \frac{f(\xb + \epsilon\cdot \bm{e}_i) -
    f(\xb)}{\epsilon}, 
    $$
    \item Symmetric difference quotient for $\eps > 0$: 
    $$
    D_\xb(\epsilon) = \frac{f(\xb + \epsilon\cdot \bm{e}_i) - f(\xb - \epsilon
    \cdot \bm{e}_{i})}{2\epsilon}
    $$
  \end{itemize}
  \item Symmetric approximation is more accurate, but $f$ must be evaluated twice.
  % , da $f(x)$ in der Regel ja schon bekannt ist.
  \item \textbf{Essential question:} How should $\epsilon$ be chosen? In any case we require $\epsilon > \epsilon_{m}$ ($\eps$ should be larger than the machine epsilon).
  \end{itemize}
  
  \framebreak
  
  From now on, let us consider univariate functions $f: \R \to \R$ only. We calculate the Taylor series at the location $\epsilon = 0$ with
  
  $$
    f(x + \epsilon) = \sum_{k=0}^\infty \frac{\epsilon^k}{k!} \nabla^{(k)} f(x):
  $$
  
  \begin{eqnarray*}
  \text{(A)} \qquad f(x + \epsilon) &=& f(x) + f'(x) \cdot \epsilon + \frac{1}{2 !}f''(x)\cdot\epsilon^2 + \frac{1}{3 !}f^{(3)}(x)\cdot\epsilon^3 \\ &+& \frac{1}{4 !}f^{(4)}(x)\cdot\epsilon^4 + ... \\
  \text{(B)} \qquad  f(x - \epsilon) &=& f(x) - f'(x) \cdot \epsilon + \frac{1}{2 !}f''(x)\cdot\epsilon^2 - \frac{1}{3 !}f^{(3)}(x)\cdot\epsilon^3 \\ &+& \frac{1}{4 !}f^{(4)}(x)\cdot\epsilon^4 - ...
  \end{eqnarray*}
  
  \framebreak
  
  Thus the following applies for Newton's difference quotient
  
  \begin{footnotesize}
  \begin{eqnarray*}
  \frac{f(x + \epsilon) - f(x)}{\epsilon} &\approx&
  f\!\,'(x) +
  \underbrace{f\!\,''(x)\frac{\epsilon}{2} + \frac{1}{3 !}f^{(3)}(x)\cdot\epsilon^2 + ... }_{\text{Error } \in \order(\epsilon)}
  \end{eqnarray*}
  \end{footnotesize}
  
  and for the symmetric difference quotient
  
  \begin{footnotesize}
  \begin{eqnarray*}
  \frac{f(x + \epsilon) - f(x - \epsilon)}{2 \epsilon} &\approx& \frac{\text{(A)} - \text{(B)} }{2 \epsilon} = \frac{1}{2 \epsilon} \left(2\cdot f\!\,'(x) \epsilon + 2 \frac{1}{3 !}f^{(3)}(x)\cdot\epsilon^3 + 2  \frac{1}{5 !}f^{(5)}(x)\epsilon^5  \right) \\ &=&
  f\!\,'(x) +
  \underbrace{c_1\epsilon^2 + c_2 \epsilon^4 + ... }_{\text{Error } \in \order(\epsilon^2)} \\
  \end{eqnarray*}
  \end{footnotesize}
  
  with $c_1 = \frac{1}{3 !}f^{(3)}(x)$ and $c_2 = \frac{1}{5 !}f^{(5)}(x)$.
  
  \framebreak
  
  We observe a \textbf{trade off} between the mathematical and a numerical error:
  \begin{itemize}
  \item The \textbf{mathematical error} is smaller for smaller $\epsilon$:
  \begin{itemize}
  \item The error is in $\order(\epsilon)$ for Newton's difference quotient
  \item The error is in $\order(\epsilon^2)$ for the Symmetric difference quotient (so symmetrical is much more accurate)
  \end{itemize}
  \item The \textbf{numerical error} error may explode for small $\eps$: The problem is extremely bad conditioned for small $\epsilon$ (loss of significance and then division by $\epsilon$).
  \end{itemize}
  
  \lz
  
  \textbf{Aim}: When choosing $\epsilon$, we have to find a compromise between mathematical and numerical error!
  
  \framebreak
  
  Let $\delta$ be a bound for the relative error in the calculation of
  $f(x)$ and $f(x+\epsilon)$ (that is, we only have access to $\tilde f(x), \tilde f(x + \epsilon)$ with relative error $\delta$).
  
  \lz
  
  We estimate the error of the numerical differentiation:
  \begin{footnotesize}
  \begin{eqnarray*}
  \Bigl|\underbrace{\frac{\tilde f(x+\epsilon) - \tilde f(x)}{\epsilon}}_{\text{our approach}} - \underbrace{f\!\,'(x)}_{\text{true value}} \Bigr| &\overset{(*)}{\le}& \left| f\!\,''(\zeta)\right|\frac{\epsilon}{2} + 2 \delta\left| f(x) \right|\frac 1\epsilon
  =: \frac{a\cdot\epsilon}2 + \frac{2b}\epsilon, \\
  \end{eqnarray*}
  \end{footnotesize}
  
  and minimize it by differentiation with respect to $\epsilon$:
  
  \vspace*{-0.5cm}
  
  \begin{footnotesize}
  \begin{eqnarray*}
  &\frac{a\cdot\epsilon}2 + \frac{2b}\epsilon \rightarrow
  \underset\epsilon {\min} \ \Leftrightarrow \ \frac a2 - \frac
  {2b}{\epsilon^2} = 0 \ \Leftrightarrow \ \frac a2 = \frac
  {2b}{\epsilon^2} \ \Leftrightarrow \ \epsilon^2=\frac{4b}a\\
  \\
  &\epsilon = 2 \sqrt{\frac ba} = 2
  \sqrt{\frac{\delta\left|f(x)\right|}{\left| f\!\,''(\zeta)\right|}}
  \end{eqnarray*}
  \end{footnotesize}
  
  
  \framebreak
  
  \begin{footnotesize}
  
  $^{(*)}$ Proof: Since we only have access to approximate values $\tilde f(x + \epsilon)$, $\tilde f(x)$ with relative error $\delta$, the following applies by definition (relative error):
  
  $$\underbrace{\left|\frac{\tilde f(x + \epsilon) - f(x + \epsilon)}{f(x + \epsilon)}\right| \le \delta}_{(1)},
  \qquad \underbrace{\left|\frac{\tilde f(x) - f(x)}{f(x)}\right| \le \delta}_{(2)}
  $$
  
  Further, by using a Taylor expansion with an exact formulation of the remainder term $R_n(x; x_0)$ (\enquote{Lagrange form}) 
  
  $$
    f(x) = \sum_{k = 0}^n \frac{f^{(k)}(x_0)}{k!} \cdot \left(x - x_0\right)^k + \underbrace{\frac{f^{(n + 1)} (\zeta)}{(n + 1)!} (x - x_0)^{n + 1}}_{R_n(x;x_0)}, \quad \zeta \in [x_0; x]
  $$
  
  we get 
  
  \vspace*{-0.5cm}
  
  \begin{eqnarray*}
    && f(x + \epsilon) = f(x) + f'(x) \cdot \epsilon + \frac{f''(\zeta)}{2} \epsilon^2, \qquad \zeta \in [0, \epsilon] \\
    &\Leftrightarrow& \underbrace{\frac{f(x + \epsilon) - f(x)}{\epsilon} - f'(x) = f''(\zeta) \frac{\epsilon}{2}}_{(3)} 
  \end{eqnarray*}
  
  \framebreak 
  
  Thus we estimate the total error as follows:
  
  \begin{eqnarray*}
  && \Bigl|\frac{\tilde f(x+\epsilon) - \tilde f(x)}{\epsilon} - f\!\,'(x) \Bigr| \\ &=&  \Bigl|\frac{f(x+\epsilon) - f(x)}{\epsilon} - f'(x) + \frac{\tilde f(x+\epsilon) - f(x + \epsilon)}{\epsilon} + \frac{f(x) - \tilde f(x)}{\epsilon}\Bigr| \\
  &\le&  \underbrace{\Bigl|\frac{f(x+\epsilon) - f(x)}{\epsilon} - f'(x)\Bigr|}_{(3)} + \underbrace{\Bigl|\frac{\tilde f(x+\epsilon) - f(x + \epsilon)}{f(x + \epsilon)\epsilon} f(x + \epsilon)\Bigr|}_{(1)} + \underbrace{\Bigl|\frac{f(x) - \tilde f(x)}{f(x)\epsilon} f(x)\Bigr|}_{(2)} \\
  &\le& \left| f\!\,''(\zeta)\right|\frac{\epsilon}{2} + \delta\left| f(x + \epsilon) \right|\frac 1\epsilon + \delta\left| f(x) \right|\frac 1\epsilon \\
  &\approx& \left| f\!\,''(\zeta)\right|\frac{\epsilon}{2} + 2 \delta\left| f(x)
  \right|\frac 1\epsilon \\
  \end{eqnarray*}
  
  \end{footnotesize}
  
  \framebreak
  
  Popular choice of $\epsilon$:
  
  \begin{itemize}
  \item $\epsilon \approx \sqrt{\delta}$ (if $\left|f(x)\right| \approx \left|
  f\!\,''(\zeta)\right|$ can be assumed) or
  \item $\epsilon = \left|x\right|\sqrt{\delta}$ \\
  (For partial deriviatives $\frac{\partial f}{\partial x_i}$ we choose 
  $\left|x_i\right|\sqrt{\delta}$ analogously.)
  \end{itemize}
  
  
  Without further knowledge it is often assumed:
  
  $$
  \delta \approx \epsilon_m
  $$
  
  \lz
  
  \begin{footnotesize}
  More on the choice of $\epsilon$ and numerical differentiation: W. Press et al., \emph{Numerical Recipes}, Chapter 5.7
  \end{footnotesize}
  
  % \framebreak
  %
  % Hesse-Matrix durch:
  % \begin{itemize}
  % \item $f$ 2 mal numerisch ableiten.
  % \item $\nabla f$ 1 mal numerisch ableiten (falls $\nabla f$ bekannt).
  % \item bei Quasi-Newton aus Näherung im Algorithmus (siehe später).
  % \end{itemize}
  
  % \framebreak
  
  % <<size = "scriptsize">>=
  % grad(x = 3, y = 1.5, FUN = foo, type = "centered")
  % x = seq(0, 6, length = 100)
  % g1 = grad(x = x, y = 1.5, FUN = foo)[, "x"]
  % g2 = grad(x = x, y = 4, FUN = foo)[, "x"]
  % plot(g1 ~ x, type = "l", ylab = "f'(x, y)")
  % lines(g2 ~ x, col = "blue")
  % @
  
  \end{vbframe}
  
  \begin{vbframe}{Derivative-based vs. Derivative-free Optimization }
  
  \begin{itemize}
    \item If an objective function is assumed to be smooth and (good approximations to) derivatives are easy to compute, derivative-based optimization methods (i.e., methods requiring gradient-information) may be useful and usually yield fast convergence. 
    \item However, if for example
    \begin{itemize}
      \item smoothness cannot be assumed
      \item the problem of computing derivatives is extremely bad conditionned
      \item $f$ is time-consuming to evaluate, or 
      \item $f$ is in some way noisy
    \end{itemize}
    optimization methods that do not require any derivative information have advantages. 
  \footnotesize	
    \item Those methods are referred to as \textbf{derivative-free algorithms}, and will be covered in later chapters. 
  \end{itemize}
  
  \end{vbframe}
  
  
  % \begin{vbframe}{Richardson extrapolation}
  
  % \textbf{Idea}: Calculate \textbf{symmetric} formula $D_x(\epsilon)$ for two values of $\epsilon$ and combine both results.
  
  % \begin{itemize}
  % \item We have shown that
  
  % \begin{footnotesize}
  % \begin{eqnarray*}
  % D_x(\epsilon) &=& \frac{f(x + \epsilon) - f(x - \epsilon)}{2 \epsilon} \\ &=&
  % f\!\,'(x) + c_1\epsilon^2 + \order(\epsilon^4)
  % \end{eqnarray*}
  % \end{footnotesize}
  
  % \item We calculate the expression for $2\epsilon$
  
  % \begin{footnotesize}
  % \begin{eqnarray*}
  % D_x(2\epsilon) &=& \frac{f(x + 2\epsilon) - f(x - 2\epsilon)}{4 \epsilon} \\ &=&
  % f\!\,'(x) + c_1 4 \epsilon^2 + \order(\epsilon^4)
  % \end{eqnarray*}
  % \end{footnotesize}
  
  % \end{itemize}
  
  % We subtract the expressions and divide by $3$:
  
  % \begin{eqnarray*}
  % \frac{D_x(2\epsilon) - D_x(\epsilon)}{3} &=& c_1 \epsilon^2 + \order(\epsilon^4) \\
  % \end{eqnarray*}
  
  % Since $c_1 \epsilon^2 = D_x(\epsilon) - f'(x) + \order(\epsilon^4)$, we plug it in and obtain
  
  % \begin{eqnarray*}
  % \frac{D_x(2\epsilon) - D_x(\epsilon)}{3} &=& D_x(\epsilon) - f'(x) + \order(\epsilon^4).
  % \end{eqnarray*}
  
  % Rearranging yields
  
  % \begin{eqnarray*}
  % f'(x) &=& D_x(\epsilon) + \frac{ D_x(\epsilon) - D_x(2\epsilon)}{3} + \order(\epsilon^4) \\
  % &=& \frac{4 D_x(\epsilon) - D_x(2\epsilon)}{3} + \order(\epsilon^4)
  % \end{eqnarray*}
  
  % The last formula for calculating the derivative is known as \textbf{Richardson extrapolation}. For fixed $\epsilon$, the expression has
  
  % \begin{itemize}
  % \item Approximately the same numerical properties as the symmetric formula,
  % \item But a mathematical one of only $\order(\epsilon^4)$ (instead of $\order(\epsilon^2)$)
  % \item However, more function evaluations are required since $D_x(\epsilon)$ and $D_x(2\epsilon)$ must be calculated.
  % \end{itemize}
  
  
  % \end{vbframe}
  
  \endlecture
  \end{document}
  
  
  
  