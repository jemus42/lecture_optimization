%
\begin{enumerate}
	%
  \item The gradient $\nabla f(\mathbf{x}) = (2x_1 + x_2, x_2 + x_1)^\top$ is continuous $\Rightarrow f \in \mathcal{C}^1$.
	\item The direction of greatest increase is given by the gradient, i.e., $\nabla f(1,1) = (3,2)^\top.$
	\item Let $\mathbf{v} \in \R^2$ be a direction with fixed length $\Vert \mathbf{v}\Vert_2 = r > 0$. \newline The directional derivative $D_{\mathbf{v}}f(\mathbf{x}) = \nabla f(\mathbf{x})^\top \mathbf{v} = \left\Vert \nabla f(\mathbf{x}) \right\Vert_2 \left\Vert \mathbf{v} \right\Vert_2 \cos (\theta) = \left\Vert \nabla f(\mathbf{x}) \right\Vert_2 r \cos (\theta).$ This becomes minimal if $\theta = \pi$, i.e., if $\mathbf{v}$ points in the opposite direction of $\nabla f \Rightarrow \mathbf{v} = -\nabla f(\mathbf{x})$ if $r = \left\Vert \nabla f (\mathbf{x}) \right\Vert_2.$ 
	Here, the direction of greatest decrease is given by $-\nabla f(1,1) = (-3, -2)^\top$.
	\item $D_{\mathbf{v}}f(\mathbf{x}) = \nabla f(1, 1)^\top \mathbf{v} \overset{!}{=} 0 \Rightarrow (3,2) \cdot \mathbf{v} = 0 \iff \mathbf{v} = \alpha \cdot (-2, 3)^\top$ with $\alpha \in \R$ and $\alpha \neq 0.$
	
	\item When we differentiate both sides of the equation $f(\tilde{\mathbf{x}}(t)) = f(1,1)$  w.r.t. $t$ we arrive at $\frac{d}{dt} f(\tilde{\mathbf{x}}(t)) = 0$.
	Via the chain rule it follows that $\nabla f(\tilde{\mathbf{x}})^\top \frac{d}{dt}\tilde{\mathbf{x}} = 0.$
	\item The gradient is orthogonal to the tangent line of the level curves.
	  
\end{enumerate}
