%\documentclass{beamer}
\documentclass[11pt,compress,t,notes=noshow, xcolor=table]{beamer}

\usepackage[orientation=landscape,size=a0,scale=1.4,debug]{beamerposter}
\mode<presentation>{\usetheme{mlr}}

\usepackage[utf8]{inputenc} % UTF-8
\usepackage[english]{babel} % Language
\usepackage{hyperref} % Hyperlinks
\usepackage{ragged2e} % Text position
\usepackage[export]{adjustbox} % Image position
\usepackage[most]{tcolorbox}
%\usepackage{nomencl}
%\makenomenclature
\usepackage{amsmath}
\usepackage{bm}
\usepackage{dsfont}
\usepackage{verbatim}
\usepackage{amsfonts}
\usepackage{csquotes}
\usepackage{multirow}
\usepackage{longtable}
\usepackage{enumerate}
\usepackage[absolute,overlay]{textpos}
\usepackage{psfrag}
\usepackage{algorithm}
\usepackage{algpseudocode}
\usepackage{eqnarray}
\usepackage{arydshln}
\usepackage{tabularx}
\usepackage{placeins}
\usepackage{tikz}
\usepackage{setspace}
\usepackage{colortbl}
\usepackage{mathtools}
\usepackage{wrapfig}



% math spaces
\ifdefined\N                                                                
\renewcommand{\N}{\mathds{N}} % N, naturals
\else \newcommand{\N}{\mathds{N}} \fi 
\newcommand{\Z}{\mathds{Z}} % Z, integers
\newcommand{\Q}{\mathds{Q}} % Q, rationals
\newcommand{\R}{\mathds{R}} % R, reals
\ifdefined\C 
  \renewcommand{\C}{\mathds{C}} % C, complex
\else \newcommand{\C}{\mathds{C}} \fi
\newcommand{\continuous}{\mathcal{C}} % C, space of continuous functions
\newcommand{\M}{\mathcal{M}} % machine numbers
\newcommand{\epsm}{\epsilon_m} % maximum error

% counting / finite sets
\newcommand{\setzo}{\{0, 1\}} % set 0, 1
\newcommand{\setmp}{\{-1, +1\}} % set -1, 1
\newcommand{\unitint}{[0, 1]} % unit interval

% basic math stuff
\newcommand{\xt}{\tilde x} % x tilde
\newcommand{\argmax}{\operatorname{arg\,max}} % argmax
\newcommand{\argmin}{\operatorname{arg\,min}} % argmin
\newcommand{\argminlim}{\mathop{\mathrm{arg\,min}}\limits} % argmax with limits
\newcommand{\argmaxlim}{\mathop{\mathrm{arg\,max}}\limits} % argmin with limits  
\newcommand{\sign}{\operatorname{sign}} % sign, signum
\newcommand{\I}{\mathbb{I}} % I, indicator
\newcommand{\order}{\mathcal{O}} % O, order
\newcommand{\pd}[2]{\frac{\partial{#1}}{\partial #2}} % partial derivative
\newcommand{\floorlr}[1]{\left\lfloor #1 \right\rfloor} % floor
\newcommand{\ceillr}[1]{\left\lceil #1 \right\rceil} % ceiling

% sums and products
\newcommand{\sumin}{\sum\limits_{i=1}^n} % summation from i=1 to n
\newcommand{\sumim}{\sum\limits_{i=1}^m} % summation from i=1 to m
\newcommand{\sumjn}{\sum\limits_{j=1}^n} % summation from j=1 to p
\newcommand{\sumjp}{\sum\limits_{j=1}^p} % summation from j=1 to p
\newcommand{\sumik}{\sum\limits_{i=1}^k} % summation from i=1 to k
\newcommand{\sumkg}{\sum\limits_{k=1}^g} % summation from k=1 to g
\newcommand{\sumjg}{\sum\limits_{j=1}^g} % summation from j=1 to g
\newcommand{\meanin}{\frac{1}{n} \sum\limits_{i=1}^n} % mean from i=1 to n
\newcommand{\meanim}{\frac{1}{m} \sum\limits_{i=1}^m} % mean from i=1 to n
\newcommand{\meankg}{\frac{1}{g} \sum\limits_{k=1}^g} % mean from k=1 to g
\newcommand{\prodin}{\prod\limits_{i=1}^n} % product from i=1 to n
\newcommand{\prodkg}{\prod\limits_{k=1}^g} % product from k=1 to g
\newcommand{\prodjp}{\prod\limits_{j=1}^p} % product from j=1 to p

% linear algebra
\newcommand{\one}{\boldsymbol{1}} % 1, unitvector
\newcommand{\zero}{\mathbf{0}} % 0-vector
\newcommand{\id}{\boldsymbol{I}} % I, identity
\newcommand{\diag}{\operatorname{diag}} % diag, diagonal
\newcommand{\trace}{\operatorname{tr}} % tr, trace
\newcommand{\spn}{\operatorname{span}} % span
\newcommand{\scp}[2]{\left\langle #1, #2 \right\rangle} % <.,.>, scalarproduct
\newcommand{\mat}[1]{\begin{pmatrix} #1 \end{pmatrix}} % short pmatrix command
\newcommand{\Amat}{\mathbf{A}} % matrix A
\newcommand{\Deltab}{\mathbf{\Delta}} % error term for vectors

% basic probability + stats
\renewcommand{\P}{\mathds{P}} % P, probability
\newcommand{\E}{\mathds{E}} % E, expectation
\newcommand{\var}{\mathsf{Var}} % Var, variance
\newcommand{\cov}{\mathsf{Cov}} % Cov, covariance
\newcommand{\corr}{\mathsf{Corr}} % Corr, correlation
\newcommand{\normal}{\mathcal{N}} % N of the normal distribution
\newcommand{\iid}{\overset{i.i.d}{\sim}} % dist with i.i.d superscript
\newcommand{\distas}[1]{\overset{#1}{\sim}} % ... is distributed as ...

% machine learning
\newcommand{\Xspace}{\mathcal{X}} % X, input space
\newcommand{\Yspace}{\mathcal{Y}} % Y, output space
\newcommand{\nset}{\{1, \ldots, n\}} % set from 1 to n
\newcommand{\pset}{\{1, \ldots, p\}} % set from 1 to p
\newcommand{\gset}{\{1, \ldots, g\}} % set from 1 to g
\newcommand{\Pxy}{\mathbb{P}_{xy}} % P_xy
\newcommand{\Exy}{\mathbb{E}_{xy}} % E_xy: Expectation over random variables xy
\newcommand{\xv}{\mathbf{x}} % vector x (bold)
\newcommand{\xtil}{\tilde{\mathbf{x}}} % vector x-tilde (bold)
\newcommand{\yv}{\mathbf{y}} % vector y (bold)
\newcommand{\xy}{(\xv, y)} % observation (x, y)
\newcommand{\xvec}{\left(x_1, \ldots, x_p\right)^\top} % (x1, ..., xp) 
\newcommand{\Xmat}{\mathbf{X}} % Design matrix
\newcommand{\allDatasets}{\mathds{D}} % The set of all datasets
\newcommand{\allDatasetsn}{\mathds{D}_n}  % The set of all datasets of size n 
\newcommand{\D}{\mathcal{D}} % D, data
\newcommand{\Dn}{\D_n} % D_n, data of size n
\newcommand{\Dtrain}{\mathcal{D}_{\text{train}}} % D_train, training set
\newcommand{\Dtest}{\mathcal{D}_{\text{test}}} % D_test, test set
\newcommand{\xyi}[1][i]{\left(\xv^{(#1)}, y^{(#1)}\right)} % (x^i, y^i), i-th observation
\newcommand{\Dset}{\left( \xyi[1], \ldots, \xyi[n]\right)} % {(x1,y1)), ..., (xn,yn)}, data
\newcommand{\defAllDatasetsn}{(\Xspace \times \Yspace)^n} % Def. of the set of all datasets of size n 
\newcommand{\defAllDatasets}{\bigcup_{n \in \N}(\Xspace \times \Yspace)^n} % Def. of the set of all datasets 
\newcommand{\xdat}{\left\{ \xv^{(1)}, \ldots, \xv^{(n)}\right\}} % {x1, ..., xn}, input data
\newcommand{\ydat}{\left\{ \yv^{(1)}, \ldots, \yv^{(n)}\right\}} % {y1, ..., yn}, input data
\newcommand{\yvec}{\left(y^{(1)}, \hdots, y^{(n)}\right)^\top} % (y1, ..., yn), vector of outcomes
\renewcommand{\xi}[1][i]{\xv^{(#1)}} % x^i, i-th observed value of x
\newcommand{\yi}[1][i]{y^{(#1)}} % y^i, i-th observed value of y 
\newcommand{\xivec}{\left(x^{(i)}_1, \ldots, x^{(i)}_p\right)^\top} % (x1^i, ..., xp^i), i-th observation vector
\newcommand{\xj}{\xv_j} % x_j, j-th feature
\newcommand{\xjvec}{\left(x^{(1)}_j, \ldots, x^{(n)}_j\right)^\top} % (x^1_j, ..., x^n_j), j-th feature vector
\newcommand{\phiv}{\mathbf{\phi}} % Basis transformation function phi
\newcommand{\phixi}{\mathbf{\phi}^{(i)}} % Basis transformation of xi: phi^i := phi(xi)

%%%%%% ml - models general
\newcommand{\lamv}{\bm{\lambda}} % lambda vector, hyperconfiguration vector
\newcommand{\Lam}{\bm{\Lambda}}	 % Lambda, space of all hpos
% Inducer / Inducing algorithm
\newcommand{\preimageInducer}{\left(\defAllDatasets\right)\times\Lam} % Set of all datasets times the hyperparameter space
\newcommand{\preimageInducerShort}{\allDatasets\times\Lam} % Set of all datasets times the hyperparameter space
% Inducer / Inducing algorithm
\newcommand{\ind}{\mathcal{I}} % Inducer, inducing algorithm, learning algorithm 

% continuous prediction function f
\newcommand{\ftrue}{f_{\text{true}}}  % True underlying function (if a statistical model is assumed)
\newcommand{\ftruex}{\ftrue(\xv)} % True underlying function (if a statistical model is assumed)
\newcommand{\fx}{f(\xv)} % f(x), continuous prediction function
\newcommand{\fdomains}{f: \Xspace \rightarrow \R^g} % f with domain and co-domain
\newcommand{\Hspace}{\mathcal{H}} % hypothesis space where f is from
\newcommand{\fbayes}{f^{\ast}} % Bayes-optimal model
\newcommand{\fxbayes}{f^{\ast}(\xv)} % Bayes-optimal model
\newcommand{\fkx}[1][k]{f_{#1}(\xv)} % f_j(x), discriminant component function
\newcommand{\fh}{\hat{f}} % f hat, estimated prediction function
\newcommand{\fxh}{\fh(\xv)} % fhat(x)
\newcommand{\fxt}{f(\xv ~|~ \thetab)} % f(x | theta)
\newcommand{\fxi}{f\left(\xv^{(i)}\right)} % f(x^(i))
\newcommand{\fxih}{\hat{f}\left(\xv^{(i)}\right)} % f(x^(i))
\newcommand{\fxit}{f\left(\xv^{(i)} ~|~ \thetab\right)} % f(x^(i) | theta)
\newcommand{\fhD}{\fh_{\D}} % fhat_D, estimate of f based on D
\newcommand{\fhDtrain}{\fh_{\Dtrain}} % fhat_Dtrain, estimate of f based on D
\newcommand{\fhDnlam}{\fh_{\Dn, \lamv}} %model learned on Dn with hp lambda
\newcommand{\fhDlam}{\fh_{\D, \lamv}} %model learned on D with hp lambda
\newcommand{\fhDnlams}{\fh_{\Dn, \lamv^\ast}} %model learned on Dn with optimal hp lambda 
\newcommand{\fhDlams}{\fh_{\D, \lamv^\ast}} %model learned on D with optimal hp lambda 

% discrete prediction function h
\newcommand{\hx}{h(\xv)} % h(x), discrete prediction function
\newcommand{\hh}{\hat{h}} % h hat
\newcommand{\hxh}{\hat{h}(\xv)} % hhat(x)
\newcommand{\hxt}{h(\xv | \thetab)} % h(x | theta)
\newcommand{\hxi}{h\left(\xi\right)} % h(x^(i))
\newcommand{\hxit}{h\left(\xi ~|~ \thetab\right)} % h(x^(i) | theta)
\newcommand{\hbayes}{h^{\ast}} % Bayes-optimal classification model
\newcommand{\hxbayes}{h^{\ast}(\xv)} % Bayes-optimal classification model

% yhat
\newcommand{\yh}{\hat{y}} % yhat for prediction of target
\newcommand{\yih}{\hat{y}^{(i)}} % yhat^(i) for prediction of ith targiet
\newcommand{\resi}{\yi- \yih}

% theta
\newcommand{\thetah}{\hat{\theta}} % theta hat
\newcommand{\thetab}{\bm{\theta}} % theta vector
\newcommand{\thetabh}{\bm{\hat\theta}} % theta vector hat
\newcommand{\thetat}[1][t]{\thetab^{[#1]}} % theta^[t] in optimization
\newcommand{\thetatn}[1][t]{\thetab^{[#1 +1]}} % theta^[t+1] in optimization
\newcommand{\thetahDnlam}{\thetabh_{\Dn, \lamv}} %theta learned on Dn with hp lambda
\newcommand{\thetahDlam}{\thetabh_{\D, \lamv}} %theta learned on D with hp lambda
\newcommand{\mint}{\min_{\thetab \in \Theta}} % min problem theta
\newcommand{\argmint}{\argmin_{\thetab \in \Theta}} % argmin theta

% densities + probabilities
% pdf of x 
\newcommand{\pdf}{p} % p
\newcommand{\pdfx}{p(\xv)} % p(x)
\newcommand{\pixt}{\pi(\xv~|~ \thetab)} % pi(x|theta), pdf of x given theta
\newcommand{\pixit}[1][i]{\pi\left(\xi[#1] ~|~ \thetab\right)} % pi(x^i|theta), pdf of x given theta
\newcommand{\pixii}[1][i]{\pi\left(\xi[#1]\right)} % pi(x^i), pdf of i-th x 

% pdf of (x, y)
\newcommand{\pdfxy}{p(\xv,y)} % p(x, y)
\newcommand{\pdfxyt}{p(\xv, y ~|~ \thetab)} % p(x, y | theta)
\newcommand{\pdfxyit}{p\left(\xi, \yi ~|~ \thetab\right)} % p(x^(i), y^(i) | theta)

% pdf of x given y
\newcommand{\pdfxyk}[1][k]{p(\xv | y= #1)} % p(x | y = k)
\newcommand{\lpdfxyk}[1][k]{\log p(\xv | y= #1)} % log p(x | y = k)
\newcommand{\pdfxiyk}[1][k]{p\left(\xi | y= #1 \right)} % p(x^i | y = k)

% prior probabilities
\newcommand{\pik}[1][k]{\pi_{#1}} % pi_k, prior
\newcommand{\lpik}[1][k]{\log \pi_{#1}} % log pi_k, log of the prior
\newcommand{\pit}{\pi(\thetab)} % Prior probability of parameter theta

% posterior probabilities
\newcommand{\post}{\P(y = 1 ~|~ \xv)} % P(y = 1 | x), post. prob for y=1
\newcommand{\postk}[1][k]{\P(y = #1 ~|~ \xv)} % P(y = k | y), post. prob for y=k
\newcommand{\pidomains}{\pi: \Xspace \rightarrow \unitint} % pi with domain and co-domain
\newcommand{\pibayes}{\pi^{\ast}} % Bayes-optimal classification model
\newcommand{\pixbayes}{\pi^{\ast}(\xv)} % Bayes-optimal classification model
\newcommand{\pix}{\pi(\xv)} % pi(x), P(y = 1 | x)
\newcommand{\piv}{\bm{\pi}} % pi, bold, as vector
\newcommand{\pikx}[1][k]{\pi_{#1}(\xv)} % pi_k(x), P(y = k | x)
\newcommand{\pikxt}[1][k]{\pi_{#1}(\xv ~|~ \thetab)} % pi_k(x | theta), P(y = k | x, theta)
\newcommand{\pixh}{\hat \pi(\xv)} % pi(x) hat, P(y = 1 | x) hat
\newcommand{\pikxh}[1][k]{\hat \pi_{#1}(\xv)} % pi_k(x) hat, P(y = k | x) hat
\newcommand{\pixih}{\hat \pi(\xi)} % pi(x^(i)) with hat
\newcommand{\pikxih}[1][k]{\hat \pi_{#1}(\xi)} % pi_k(x^(i)) with hat
\newcommand{\pdfygxt}{p(y ~|~\xv, \thetab)} % p(y | x, theta)
\newcommand{\pdfyigxit}{p\left(\yi ~|~\xi, \thetab\right)} % p(y^i |x^i, theta)
\newcommand{\lpdfygxt}{\log \pdfygxt } % log p(y | x, theta)
\newcommand{\lpdfyigxit}{\log \pdfyigxit} % log p(y^i |x^i, theta)

% probababilistic
\newcommand{\bayesrulek}[1][k]{\frac{\P(\xv | y= #1) \P(y= #1)}{\P(\xv)}} % Bayes rule
\newcommand{\muk}{\bm{\mu_k}} % mean vector of class-k Gaussian (discr analysis) 

% residual and margin
\newcommand{\eps}{\epsilon} % residual, stochastic
\newcommand{\epsi}{\epsilon^{(i)}} % epsilon^i, residual, stochastic
\newcommand{\epsh}{\hat{\epsilon}} % residual, estimated
\newcommand{\yf}{y \fx} % y f(x), margin
\newcommand{\yfi}{\yi \fxi} % y^i f(x^i), margin
\newcommand{\Sigmah}{\hat \Sigma} % estimated covariance matrix
\newcommand{\Sigmahj}{\hat \Sigma_j} % estimated covariance matrix for the j-th class

% ml - loss, risk, likelihood
\newcommand{\Lyf}{L\left(y, f\right)} % L(y, f), loss function
\newcommand{\Lypi}{L\left(y, \pi\right)} % L(y, pi), loss function
\newcommand{\Lxy}{L\left(y, \fx\right)} % L(y, f(x)), loss function
\newcommand{\Lxyi}{L\left(\yi, \fxi\right)} % loss of observation
\newcommand{\Lxyt}{L\left(y, \fxt\right)} % loss with f parameterized
\newcommand{\Lxyit}{L\left(\yi, \fxit\right)} % loss of observation with f parameterized
\newcommand{\Lxym}{L\left(\yi, f\left(\bm{\tilde{x}}^{(i)} ~|~ \thetab\right)\right)} % loss of observation with f parameterized
\newcommand{\Lpixy}{L\left(y, \pix\right)} % loss in classification
\newcommand{\Lpiv}{L\left(y, \piv\right)} % loss in classification
\newcommand{\Lpixyi}{L\left(\yi, \pixii\right)} % loss of observation in classification
\newcommand{\Lpixyt}{L\left(y, \pixt\right)} % loss with pi parameterized
\newcommand{\Lpixyit}{L\left(\yi, \pixit\right)} % loss of observation with pi parameterized
\newcommand{\Lhxy}{L\left(y, \hx\right)} % L(y, h(x)), loss function on discrete classes
\newcommand{\Lr}{L\left(r\right)} % L(r), loss defined on residual (reg) / margin (classif)
\newcommand{\lone}{|y - \fx|} % L1 loss
\newcommand{\ltwo}{\left(y - \fx\right)^2} % L2 loss
\newcommand{\lbernoullimp}{\ln(1 + \exp(-y \cdot \fx))} % Bernoulli loss for -1, +1 encoding
\newcommand{\lbernoullizo}{- y \cdot \fx + \log(1 + \exp(\fx))} % Bernoulli loss for 0, 1 encoding
\newcommand{\lcrossent}{- y \log \left(\pix\right) - (1 - y) \log \left(1 - \pix\right)} % cross-entropy loss
\newcommand{\lbrier}{\left(\pix - y \right)^2} % Brier score
\newcommand{\risk}{\mathcal{R}} % R, risk
\newcommand{\riskbayes}{\mathcal{R}^\ast}
\newcommand{\riskf}{\risk(f)} % R(f), risk
\newcommand{\riskdef}{\E_{y|\xv}\left(\Lxy \right)} % risk def (expected loss)
\newcommand{\riskt}{\mathcal{R}(\thetab)} % R(theta), risk
\newcommand{\riske}{\mathcal{R}_{\text{emp}}} % R_emp, empirical risk w/o factor 1 / n
\newcommand{\riskeb}{\bar{\mathcal{R}}_{\text{emp}}} % R_emp, empirical risk w/ factor 1 / n
\newcommand{\riskef}{\riske(f)} % R_emp(f)
\newcommand{\risket}{\mathcal{R}_{\text{emp}}(\thetab)} % R_emp(theta)
\newcommand{\riskr}{\mathcal{R}_{\text{reg}}} % R_reg, regularized risk
\newcommand{\riskrt}{\mathcal{R}_{\text{reg}}(\thetab)} % R_reg(theta)
\newcommand{\riskrf}{\riskr(f)} % R_reg(f)
\newcommand{\riskrth}{\hat{\mathcal{R}}_{\text{reg}}(\thetab)} % hat R_reg(theta)
\newcommand{\risketh}{\hat{\mathcal{R}}_{\text{emp}}(\thetab)} % hat R_emp(theta)
\newcommand{\LL}{\mathcal{L}} % L, likelihood
\newcommand{\LLt}{\mathcal{L}(\thetab)} % L(theta), likelihood
\newcommand{\LLtx}{\mathcal{L}(\thetab | \xv)} % L(theta|x), likelihood
\newcommand{\logl}{\ell} % l, log-likelihood
\newcommand{\loglt}{\logl(\thetab)} % l(theta), log-likelihood
\newcommand{\logltx}{\logl(\thetab | \xv)} % l(theta|x), log-likelihood
\newcommand{\errtrain}{\text{err}_{\text{train}}} % training error
\newcommand{\errtest}{\text{err}_{\text{test}}} % test error
\newcommand{\errexp}{\overline{\text{err}_{\text{test}}}} % avg training error

% lm
\newcommand{\thx}{\thetab^\top \xv} % linear model
\newcommand{\olsest}{(\Xmat^\top \Xmat)^{-1} \Xmat^\top \yv} % OLS estimator in LM 

{}

\title{Optimization :\,: CHEAT SHEET} % Package title in header, \, adds thin space between ::
\newcommand{\packagedescription}{ % Package description in header
	The \textbf{I2ML}: Introduction to ML course offers an introductory and applied overview of "supervised" ML. It is organized as a digital lecture.
}

\newlength{\columnheight} % Adjust depending on header height
\setlength{\columnheight}{84cm}

\newtcolorbox{codebox}{%
	sharp corners,
	leftrule=0pt,
	rightrule=0pt,
	toprule=0pt,
	bottomrule=0pt,
	hbox}

\newtcolorbox{codeboxmultiline}[1][]{%
	sharp corners,
	leftrule=0pt,
	rightrule=0pt,
	toprule=0pt,
	bottomrule=0pt,
	#1}

\begin{document}
\begin{frame}[fragile]{}
\begin{columns}
	\begin{column}{.31\textwidth}
		\begin{beamercolorbox}[center]{postercolumn}
			\begin{minipage}{.98\textwidth}
				\parbox[t][\columnheight]{\textwidth}{
					\vspace{1cm}
					\textbf{Note: } The lecture uses many examples from maths, statistics and machine learning. Therefore notation may overlap, but the context should make clear how to uncerstand the notation. 
					If notation is unclear nevertheless, please contact the instructors.
					\begin{myblock}{General Mathematical Notation (I)}
						\begin{codebox}
							$\N, ~\Z, ~\R$ : natural, integer, and real numbers
						\end{codebox}
						\hspace*{1ex}% $\Z = \{..., -3, -2, -1, 0, 1, 2, 3, ...\}$.
						% \begin{codebox}
						%     $a \in \{\N, \Z, \R\}$ :  a (natural, integer, real) scalar value % $\pdfxyt$
						% \end{codebox}
						% \hspace*{1ex} Lowercase letters denote scalar values.
						% \\
						\begin{codebox}
						    $|a|$ : absolute value of a scalar $a \in \R$
						\end{codebox}
						\hspace*{1ex}
						\begin{codebox}
							 $\mathcal{I} = [a, b] \subseteq \R$ : interval ranging from $a \in \R$ to $b \in \R$, $a < b$
						\end{codebox}
						\hspace*{1ex}
						\begin{codebox}
						    $\xv = (x_1, x_2, ..., x_n) \in \R^n$ :  real vector in $\R^n$ % $\pdfxyt$
						\end{codebox}
						\hspace*{1ex} vectors in \textbf{lowercase bold}; refer to $i$-th element by \textbf{sub}script $i$.
						\\
						\begin{codebox}
							$\mathbf{e}_k$ : $k$-th unit vector
						\end{codebox}
						\hspace*{1ex} vector which is zero at all positions $\neq k$, and one at position $k$ \\
						\begin{codebox}
						    $\xv^{[t]}$ :  $t$-th iteration of a sequence $\left(\xv^{[1]}, \xv^{[2]}, \xv^{[3]}, ...\right)$
						\end{codebox}
						\hspace*{1ex} $t \in \{1, ..., T\}$ (finite sequence) or $t \in \N$ (infinite sequence)
						\\
						\begin{codebox}
						    $\xv^{(i)}$ :  $i$-th element of an indexed family $\left\{\xv^{(1)}, \xv^{(2)}, ..., \xv^{(n)}\right\}$
						\end{codebox}
						\hspace*{1ex} collection of vectors, indexed by indices $i$ of an index set\\
						\begin{codebox}
						    $\bm{A} = \left(A_{i,j}\right)_{1\le i \le n, 1 \le j \le m}$ :  matrix in $\R^{n \times m} $
						\end{codebox}
						\hspace*{1ex} matrices in \textbf{uppercase bold}, \textbf{sub}script $(i,j)$ denotes row $i$, column $j$
						\\
						\begin{codebox}
							$\mathbf{I}_n$ : $n \times n$ identity matrix
						\end{codebox}
						\hspace*{1ex} %Matrix that has $1$'s on the diagonal, and $0$'s on non diagnoal elements. \\
						\begin{codebox}
						    $\xv^\top \in \R^{1 \times n}, \bm{A}^\top \in \R^{m \times n}$ :  Transpose of a vector / matrix
						\end{codebox}
						\hspace*{1ex}
						\begin{codebox}
						    $\bm{A}^{-1} \in \R^{n \times n}$ :  Inverse of a quadratic matrix $\bm{A} \in \R^{n \times n}$ (if it exists)
						\end{codebox}
						\hspace*{1ex}
						\begin{codebox}
						    $\|\xv\|, \|\bm{A}\|$ : vector norm, (induced) matrix norm
						\end{codebox}
						\hspace*{1ex} %$\|.\|_1$ denotes Manhattan norm, $\|.\|_2$ Euclidean norm, $\|.\|_\infty$ maximum norm, 
						$\|.\|_p$ denotes a $p$-norm, $\|\bm{A}\|_F$ denotes the Frobenius matrix norm
						\\
						\begin{codebox}
						    $\text{det}\left(\bm{A}\right), |\bm{A}|$ : determinant of $\bm{A}$
						\end{codebox}
						\hspace*{1ex}
						\begin{codebox}
							$x \text{ mod } m$ : modulo operation (remainder after division by $m$)
						\end{codebox}
					\end{myblock}
				}
			\end{minipage}
		\end{beamercolorbox}
	\end{column}
	\begin{column}{.31\textwidth}
		\begin{beamercolorbox}[center]{postercolumn}
			\begin{minipage}{.98\textwidth}
				\parbox[t][\columnheight]{\textwidth}{


						\begin{myblock}{General Mathematical Notation (II)}
						\begin{codebox}
							 $f: \mathcal{S} \to \R^m$ : function with domain $\mathcal{S}$ and codomain $\R^m$
						\end{codebox}
						\hspace*{1ex}  % usually $\mathcal{S} \subseteq \R^d$\\% $\mathcal{S} \subseteq \R^d$ is called domain, $\R^m$ is called the codomain; \\
						\begin{codebox}
							 $f~(T)$ : image of $T \subseteq \mathcal{S}$ under $f$
						\end{codebox}
						\hspace*{1ex} range of values $y$ for which there is an $\xv \in \mathcal{S}$ such that $f(\xv) = y$. \\
						\begin{codebox}
							 $\mathcal{C}^k$ : class of $k$-times continuously differentiable functions
						\end{codebox}
						\hspace*{1ex} if $f \in \mathcal{C}^k$, the $k$-th derivative of $f$ exists and it is continuous; $\mathcal{C}^\infty$ means that the function is infinitely often continuously differentiable \\
						\begin{codebox}
							 $f^\prime, f'', f^{(3)}, ...$ : derivatives for a function $f$ (if it exists)
						\end{codebox}
						\hspace*{1ex} this notation is only valid for univariate functions $f: \mathcal{S} \to \R$, $\mathcal{S} \subseteq \R$ \\
						\begin{codebox}
							 $\nabla f(\xv) = \left(\frac{\partial f(\xv)}{\partial x_1}, \frac{\partial f(\xv)}{\partial x_2}, ..., \frac{\partial f(\xv)}{\partial x_d}\right)^\top \in \R^d$ : gradient of $f$
						\end{codebox}
						\hspace*{1ex} % this notation is used for multivariate functions \\
						\begin{codebox}
							 $D_v f(\xv)$ : directional derivative for $f$ in the direction of $v \in \R^d$
						\end{codebox}
						\hspace*{1ex} % this notation is used for multivariate functions \\
						\begin{codebox}
							 $\nabla^2 f(\xv) = \left(\frac{\partial f(\xv)}{\partial x_i \partial x_j}\right)_{i, j=1, ..., d}$ : Hessian of $f$ (if it exists)
						\end{codebox}
						\hspace*{1ex} often also denoted by $\mathbf{H}(\xv)$ \\
					\end{myblock}
						\vspace*{-2cm}
						\begin{myblock}{Probability Theory and Statistics}
						\begin{codebox}
							 $X$ : (discrete or continuous) random variable
						\end{codebox}
 						\hspace*{1ex} % random variables (RV) are denoted in uppercase letters \\
 						\hspace*{1ex}
						\begin{codebox}
							 $\thetab, \thetah \in \Theta$ : parameter $\thetab$ of a distribution and its estimate $\thetah$
						\end{codebox}
 						\hspace*{1ex}
						\begin{codebox}
							 $F_X, f_X$ : distribution/density function of a continuous RV $X$
						\end{codebox}
 						\hspace*{1ex} % A $F_x$ denotes the distribution function of a random variable $x$, $f_x$ denotes the density.
						\begin{codebox}
							 $\P(X = k)$ : density of a discrete random variable $X$
						\end{codebox}
 						\hspace*{1ex} % A $F_x$ denotes the distribution function of a random variable $x$, $f_x$ denotes the density.
						\begin{codebox}
							 $\phi$, $\Phi$ : density, distribution function of a standard normal random variable
						\end{codebox}
						\hspace*{1ex} % Subscripts $(\mu, \sigma^2)$ indicate the mean and the variance. \\
						\begin{codebox}
							 $X \sim U(a, b)$ : $X$ is sampled uniformly from the intervall $[a, b]$
						\end{codebox}
						\hspace*{1ex}
						\begin{codebox}
							 $X \sim \mathcal{N}(\mu, \sigma^2)$ : $X$ is a Gaussian RV with mean $\mu$ and variance $\sigma^2$
						\end{codebox}
						\hspace*{1ex}
						\begin{codebox}
							 $\bm{X} \sim \mathcal{N}(\bm{\mu}, \bm{\Sigma})$ : $\bm{X}$ is a Gaussian random vector
						\end{codebox}
						\hspace*{1ex} usually, $\bm{\mu}$ denotes the mean and $\Sigma$ the covariance matrix; the mean and covariance might also be denoted by $\bm{m}$ and $\bm{C}$\\
						\begin{codebox}
							 $\mathcal{L}(\thetab), \ell(\thetab): \Theta \to \R$ : likelihood function / log-likelihood function
						\end{codebox}
						\hspace*{1ex} the log likelihood is defined as $\ell(\thetab) := \log \mathcal{L}(\thetab)$		

					\end{myblock}					
					\vfill
					}
			\end{minipage}
		\end{beamercolorbox}
	\end{column}
	\begin{column}{.31\textwidth}
		\begin{beamercolorbox}[center]{postercolumn}
			\begin{minipage}{.98\textwidth}
				\parbox[t][\columnheight]{\textwidth}{
		
					\begin{myblock}{Machine Learning}
						\begin{codebox}
							$\Xspace, \Yspace$ : input space, and label / output space 
						\end{codebox}
						\hspace*{1ex}
						\begin{codebox}
							 $\D = \Dset$ : dataset
						\end{codebox}
						\hspace*{1ex} $\xi \in \Xspace$ is called input and $\yi \in \Yspace$ is called output \\
						\begin{codebox}
							 $f: \Xspace \to \R^g$ : model
						\end{codebox}
						\hspace*{1ex} usually, $f$ is parameterized by a parameter $\thetab \in \Theta$; we write $f = f(\cdot ~|~ \thetab)$ (e.g. $\fxt = \thetab^\top \xv$)\\
						\begin{codebox}
							 $\thetab, \thetah \in \Theta$ : model parameter and its estimate
						\end{codebox}
						\hspace*{1ex}
						\begin{codebox}
							 $L: \Yspace \times \R^g \to \R^+_0$ : loss function
						\end{codebox}
						\hspace*{1ex} measures the \enquote{goodness} of a prediction $\hat y$; for example, the squared loss $L(y, \fx) = (y - \fx)^2$ \\
						\begin{codebox}
							 $\risk: \Theta \to \R$ : empirical risk function
						\end{codebox}
						\hspace*{1ex} the (empirical) risk maps a parameter $\thetab$, which parameterizes the model $f(\cdot ~|~ \thetab)$, to the sum of losses
						$\risket = \sumin L(\yi, \fxit)$
					\end{myblock}
				}
			\end{minipage}
		\end{beamercolorbox}
	\end{column}
\end{columns}
\end{frame}

\newpage

\begin{frame}[fragile]{}
\begin{columns}
	\begin{column}{.31\textwidth}
		\begin{beamercolorbox}[center]{postercolumn}
			\begin{minipage}{.98\textwidth}
				\parbox[t][\columnheight]{\textwidth}{
					\begin{myblock}{Optimization (General)}
						\begin{codebox}
							$\min\limits_{\xv \in \mathcal{S}} f(\xv), f: \mathcal{S}\subseteq \R^d \to \R^m$ :
							 optimization problem
						\end{codebox}
						\hspace*{1ex} w.l.o.g. we only consider minimization \\
						\begin{codebox}
							$\xv \in \mathcal{S}$: decision variable $\xv$ in the decision space $\mathcal{S}$
						\end{codebox}
						\hspace*{1ex}
						\begin{codebox}
							 $f: \mathcal{S} \to \R^m$ : objective function with domain $\mathcal{S} \subseteq \R^d$ and codomain $\R^m$
						\end{codebox}
						\hspace*{1ex} 
						\begin{codebox}
							$\min\limits_{\thetab \in \Theta} \risket$ : empirical risk minimization (ERM) problem
						\end{codebox}
						\hspace*{1ex} \textbf{Note:} We often consider the ERM problem in ML as an example of an optimization problem. In this case, we switch from the general optimization notation to the ML notation: 
						\begin{itemize}
							\item We optimize the function $\risket$ (instead of $f$); $f$ instead denotes the ML model
							\item We optimize over $\thetab \in \Theta$ (instead over $\xv \in \mathcal{S}$) 
						\end{itemize}
						% All further notation changes accordingly. \\
						\hspace*{1ex}			
						\begin{codebox}
							 $\xv^\ast \in \argmin\limits_{\xv \in \mathcal{S}} f(\xv)$, $y^\ast = \min\limits_{\xv \in \mathcal{S}} f(\xv)$
						\end{codebox}
						\hspace*{1ex} theoretical optimum $\xv^\ast$ (also local, depending on the context), \\
                        \hspace*{1ex} corresponding optimal value $y^\ast = f(\xv^\ast)$ \\
						\begin{codebox}
							 $\hat\xv \in \mathcal{S}, \hat y \in \R$ : estimated optimum and optimal value
						\end{codebox}
						\hspace*{1ex} typically $(\hat\xv, \hat y) = \mathcal{A}(f,\mathcal{S})$ is returned by an optimization algorithm $\mathcal{A}$ \\
						\begin{codebox}
							 $\xv^{[t]} \in \mathcal{S}$ : $t$-th step of an optimizer in the decision space
						\end{codebox}
						\vspace*{2ex} \textbf{Multivariate Optimization}\\
							\begin{codebox}
								 $\alpha \in \R_+$ : step-size / learning rate
							\end{codebox}
							\hspace*{1ex}
							\begin{codebox}
								 $\bm{d} \in \R^d$ : descent direction in $\xv$
							\end{codebox}
							\hspace*{1ex}
							% \begin{codebox}
							% 	 $\bm{\nu} \in \R^d$ : Velocity
							% \end{codebox}
							% \hspace*{1ex}
							\begin{codebox}
								 $\varphi \in [0, 1]$ : momentum
							\end{codebox}
						\end{myblock}									
					}
			\end{minipage}
		\end{beamercolorbox}
	\end{column}
	\begin{column}{.31\textwidth}
		\begin{beamercolorbox}[center]{postercolumn}
			\begin{minipage}{.98\textwidth}
				\parbox[t][\columnheight]{\textwidth}{
                    \begin{myblock}{Optimization (Constrained Optimization)}
					\begin{codebox}
							$\min\limits_{\xv \in \mathcal{S}} f(\xv)$ s.t. $h(\xv) = 0, g(\xv) \le 0$ : constrained optimization problem
						\end{codebox}
						\begin{itemize}
							\item \quad $f: \mathcal{S} \to \R^m$: objective function
							\item \quad $h: \mathcal{S} \to \R^k$ : equality constraints; $k$: number of constraints\\
							\item \quad $g: \mathcal{S} \to \R^l$ : inequality constraints; $l$: number of constraints
						\end{itemize} \hspace{1ex} 
						\begin{codebox}
							 $\mathcal{L}: \mathcal{S} \times \R^k \times \R^l$ : Lagrangian
						\end{codebox}
						\hspace*{1ex} $(\xv, \bm{\alpha}, \bm{\beta}) \mapsto \mathcal{L}(\xv, \bm{\alpha}, \bm{\beta})$; $\bm{\alpha}, \bm{\beta}$ are called Lagrange multiplier \\
						\hspace*{1ex} (in constrained optimization context, but likelihood in stats context!)
					\end{myblock}
					\begin{myblock}{Optimization (Evolutionary Algorithms)}
						\begin{codebox}
							 $P$ : population (of solution candidates)
						\end{codebox}
						\hspace*{1ex} 
						\begin{codebox}
							 $\mu \in \N$ : size of a population
						\end{codebox}
						\hspace*{1ex} 
						\begin{codebox}
							 $\lambda \in \N$ : offspring size
						\end{codebox}
						\hspace*{1ex} 
						\begin{codebox}
							 $(\mu, \lambda)$-selection : survival selection strategy
						\end{codebox}
						\hspace*{1ex} the best $\mu$ individuals from $\lambda$ candidates are chosen ($\lambda \ge \mu$ required) \\
						\begin{codebox}
							 $(\mu + \lambda)$-selection : survival selection strategy
						\end{codebox}
						\hspace*{1ex} the best $\mu$ individuals are chosen from the pool of the current population of size $\mu$ and the offspring of size $\lambda$. \\
						\begin{codebox}
							 $\xv_{i:\lambda}$ : $i$-th ranked candidate
						\end{codebox}
						\hspace*{1ex} $\lambda$ solution candidates are ranked according to some criterion (e.g. by a fitness function); $\xv_{i:\lambda}$ means that this candidate has rank $i$. \\
						\begin{codebox}
							 $\bm{m}^{[g]}, \bm{C}^{[g]}, \sigma^{[g]}$ : configurations in generation $g$
						\end{codebox}
						\hspace*{1ex} superscript $[g]$ denotes the $g$-th generation \\
						\begin{codebox}
							 $\xv^{[g](k)}$ : $k$-th individual in the population in generation $g$
						\end{codebox}
						\end{myblock}
				}
			\end{minipage}
		\end{beamercolorbox}
	\end{column}
    \begin{column}{.31\textwidth}
	       	\begin{beamercolorbox}[center]{postercolumn}
			     \begin{minipage}{.98\textwidth}
				    \parbox[t][\columnheight]{\textwidth}{
                        \begin{myblock}{Optimization (Multi-Objective)}
    						\begin{codebox}
    						      $\mathcal{P}$ : Pareto set
    						\end{codebox}
    						\hspace*{1ex} Set of nondominated solutions (in the decision space $\mathcal{S}$) \\
    						\begin{codebox}
    						      $\mathcal{F}$ : Pareto front
    						\end{codebox}
    						\hspace*{1ex} image of the Pareto set $\mathcal{P}$ under a multi-objective function $f$ \\
                        \end{myblock}
                    }
			\end{minipage}
		\end{beamercolorbox}
	\end{column}
\end{columns}
\end{frame}


\end{document}